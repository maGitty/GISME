\chapter{Conclusion}
\label{ch:Conclusion}

This thesis addressed the subject of whether grid-based weather information provides a benefit in energy forecasting or not. It may not appear entirely clear that the answer is yes, as there definitely is a benefit, even though not as expected. The behaviour that has been expected, is, that \eg filtering the most populated regions and using those grid points was to improve the forecast result, but it actually worsened it. This could be observed for a forecast using the huge number of 1435 exogenous variables, one for each grid point. It turned out, that this does not only highly comprise the computation time, but also has a very negative impact on the accuracy of the forecast. Still, there is an improvement for using \eg averaged temperature data, which is a noteworthy benefit of grid-based data, as the average temperature can be computed for any desired composition of grid points. It also needs to be mentioned, that this subject has not yet been directly addressed by any of the found related works. Previous works mainly focused on the actual forecast and how to improve it, rather than which sort of data actually provides beneficial behaviour in terms of forecasting. Further research is needed in order to figure out how grid-based data can be compressed optimally, so that the number of variables is limited to a small enough amount or generalizes well enough to avoid over-fitting. Further, forecasts with an increased time scope could be used, to evaluate the temporal range for which the current weather improves forecasts of the short-term future energy demand. In the end, there is an additional assumption made, which is, that in the near future, weather may have an even higher influence on energy consumption due to the current energy transition and possibly resulting outcomes such as \gls{dsm}. This assumption emphasizes the importance of this research topic and also the practical usefulness of the results of this thesis.\\

%This is a conclusion. It is fine because it is small and nice.\\

%By evaluating the benefit of using grid-based weather information in energy forecasting
%In the end, it can be said, that in contrast to other works presented in \Cref{ch:RW}, the subject

%It needs to be clarified, that in contrast to most of the presented works, this thesis uses reanalysed data from \gls{ecmwf} as weather predictions which means, that the forecasts might behave differently from forecasts in other works as what here is assumed to be a weather forecast is more accurate than usually. This also means that results from this thesis may not exactly match results using the same procedure with real-time data.\\

%Repeat the problem and its relevance, as well as the contribution (plus quantitative results). 
%Look back at what you have written in the introduction.

%Provide an outlook for further research steps.