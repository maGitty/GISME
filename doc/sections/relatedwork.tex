\chapter{Related Work}
\label{ch:RW}

This chapter gives an overview of related work in the field of energy forecasting considering grid-based data. After describing the general approach of search, there are some sources presented ordered by degree of relation to this thesis.\\

In order to find relevant literature, arXiv\footnote{https://arxiv.org/},Google Scholar\footnote{https://scholar.google.de/} and BASE\footnote{https://www.base-search.net/} are used.\\
In the process of search, the following criteria are applied to identify relevant literature:
\begin{itemize}
  \item The title of the paper suggests that the authors work with geographic or grid-based data
  \item The title of the paper implies that the subject of the paper is being situated in the field of energy networks
  \item The title of the paper suggests that the authors aim at forecasting values
  \item The abstract or introduction of the paper suggests that the authors work with geographic or grid-based data
  \item The abstract or introduction of the paper suggests that the authors aim at forecasting or rather how forecasting is done
\end{itemize}

Subsequently, there will be two papers outlined which provide useful information for related research. After that, the papers that meet above criteria are outlined and explained regarding the used type of data, the forecast time series, applied forecasting methods and the forecast horizon.\\

There were some papers containing valuable information such as \tcite{Kaminska-Chuchmala2014}, in particular due to its subject of forecasting internet traffic load and the high correlation between internet traffic load and electricity load, as can be read in \tcite{Morley2018}. They apply \gls{ok} to spatially interpolate station-based data. Due to the high similarity between these fields, it is has potential for related subjects and also influenced this work \eg concerning further research. Furthermore it underlines the benefit of grid-based data by illustrating the necessary effort of processing station-based data. Another suitable example utilizing station-based data is \tcite{Fairley2017} which investigates marine electricity generation and critically discusses implications for electricity supply. This combines localization issues and the electricity network. Unfortunately, the aim here is not to forecast, but only to examine the problem.\\

%1 Ludwig station-based, output: electricity price, methods: arma,armax,rf,lasso, horizon 24h, Germany
%2 Salcedo grid-based, output: solar radiation, methods: cro,elm,gga,mars,svr, horizon: 24h, Australia
%3 Diagne grid-based, output: solar radiation, methods: arma,arima,cards,nn,wnn, horizon: 5min-6h, no location?!
The first group of related work utilizes grid-based, station-based or both types of data to forecast various, not necessarily electricity generation related, time series. In the first paper elaborated by \tcite{Ludwig2015}, the use of station-based weather data from \gls{dwd} for electricity price forecasting in Germany is investigated. The price history is obtained from \gls{epex}. The work does not consider \gls{nn}, but rather compares \gls{lasso} and \gls{rf} in addition to \gls{arma} and \gls{armax} models. A desirable side effect from \gls{rf} is the output of the variable importance which is useful in order to filter variables by order of their relevance. As this work has a focus on short-term forecasts, the forecast time series here is the electricity price for the next day, thus having a forecast horizon of 24 hours. Another example of this group is \tcite{Salcedo-Sanz2018}, where grid-based weather data is used to forecast solar radiation in Australia. The evaluated methods are combinations of \gls{cro}, \gls{elm}, \gls{gga}, \gls{mars}, \gls{svr} for a forecast horizon of 24 hours. Similarly, \tcite{Diagne2013} utilizes grid-based weather data for solar radiation forecasting. Different data sources are compared, specifically \gls{ecmwf}, \gls{mm5} and \gls{wrf}. The paper focuses on \gls{ar} methods including \gls{arma}, \gls{arima} and \gls{cards}, \gls{nn} and \gls{wnn} considering short time ranges from 5 min up to 6h.\\

%4 Aguiar mixed, output: ghi, methods: nn, horizon: 1-6h, Gran Canaria Island
%5 Bofinger station-based, output: solar power generation, methods: mos,idw, horizon: 24-120h, Germany
%6 Haben station-based, output: low voltage load, methods: kde,sslr,arwd,arwdy,hwt, horizon: up to 4 days, UK
%7 Alessandrini, station-based, output: wind power, methods: anen, horizon: 0-132h, Sicily(Italy)
The second group of related work forecast electricity generation, but in contrast to the first group, make use of station-based data. \Eg \tcite{Aguiar2016} utilize both, grid-based and station-based weather data to improve \gls{ghi} forecasts on Gran Canaria Island. \gls{ghi} is similar to solar radiation that is forecast in the previous paper. In order to obtain the desired results, \gls{nn} are applied. As the authors consider intra-day forecasting, the forecast horizon is limited to a range from 1 up to 6 hours in this case. \tcite{Bofinger2006} acquire data only from local weather stations to forecast solar power generation. The data is then refined with grid-based data from \gls{ecmwf} by applying \gls{mos} and \gls{idw}, spatially interpolated and then simulated for Germany in order to predict a temporal range of 24-120 hours. In a work, that has been published by \tcite{Haben2018}, station-based weather data is applied to forecast low voltage load in the United Kingdom. They implement \gls{kde}, \gls{sslr}, \gls{arwd}, \gls{arwdy} and \gls{hwt} and compare them for forecast horizons of up to 4 days. \tcite{Alessandrini2015} utilize non-gridded wind and power data from a wind farm in northern Sicily in Italy, with which they forecast generated wind power. Here, a novel approach, an \gls{anen}, which originally is used for meteorological ensemble forecasts, is applied to the data to retain a probabilistic prediction for the next 0-132 hours.\\

%8 Sperati, grid-based, output: solar power in Italy, methods: pdf,nn,vd,emos,pe, horizon: 0-72h, Italy
%9 De Felice, grid-based, output: electricity demand in Italy, methods: lr,svm, horizon: 1-2 months, Italy
%10 Davò, grid-based, output: wind power, solar irradiance, methods: pca,nn,anen, horizon: 0-72h, Sicily(Italy)
The last group of related work forecasts electricity generation with grid-based data. An application of grid-based data from \gls{ecmwf} is proposed in \tcite{Sperati2016} for solar power prediction in Italy. They implement a \gls{pdf} combined with \gls{nn},~\gls{vd},~\gls{emos} and \gls{pe}. The time series forecast includes a range of 0-72 hours. Similar to this thesis, \tcite{DeFelice2015} use grid-based data from \gls{ecmwf} to forecast the electricity demand, though for Italy. \gls{lr} and a \gls{svm} are applied. Given that power prediction is a rather complex problem, the non-linear \gls{svm} performs better than a simple \gls{lr}. The last, and therefore most relevant paper presented, is \tcite{Davo2016} who utilize grid-based wind speed data generated by applying the \gls{rams} with boundary conditions from \gls{ecmwf}. Furthermore, they acquire grid-based data of solar radiation energy per square meter as one of the two forecast time series is the solar irradiance. The data is coming from \gls{noaaesrl} and was provided for an online competition hosted by Kaggle\footnote{https://www.kaggle.com/}. Reference power data is obtained from Terna\footnote{https://www.terna.it/} as the other predicted time series is the wind power produced over Sicily. A \gls{pca} is employed, as grid-based data is even more prone to the curse of dimensionality because of the two additional dimensions. In terms of forecasting, they apply \gls{nn} and an \gls{anen}. The forecast horizon has a range of 0 to 72 hours and the output is a prediction of both, wind power and solar radiation.\\

Comparing the works above to this thesis, it is notable that none of them focuses on evaluating the benefit of grid-based data to forecast energy time series. % However, there has been related work, even though not having the exact same subject, which is evaluating how to properly include grid-based data for energy time series forecasting. As a result, these papers are employed as reference knowledge.\\
\Cref{tab:relwork} provides an overview about the mentioned related works regarding the type of the used weather data, used methods, the place of origin of the data, the forecast horizon and the forecast time series.\\

% TODO add location column?!
\begin{sidewaystable}[!ht]%
\rowcolors{2}{white}{gray!25}
\centering
\footnotesize
\begin{tabularx}{\linewidth}{llXlll}
\tablehead paper & \tablehead type of weather data & \tablehead methods & \tablehead location & \tablehead forecast horizon & \tablehead forecast time series \\\Xhline{2\arrayrulewidth}
\tcite{Ludwig2015} & station-based & \gls{arma},\gls{armax},\gls{lasso},\gls{rf} & Germany & 24h & energy prices\\
\tcite{Salcedo-Sanz2018} & grid-based & \gls{elm},\gls{cro},\gls{mars},\newline\gls{mlr},\gls{svr},\gls{gga} & Australia & 24h & solar radiation\\
\tcite{Diagne2013} & grid-based & \gls{arma},\gls{arima},\gls{cards},\newline\gls{nn},\gls{wnn} & - & 5 min-6h & solar radiation\\\Xhline{2\arrayrulewidth}
\tcite{Aguiar2016} & mixed & \gls{nn} & Gran Canaria Island & 1-6h & solar radiation\\
\tcite{Bofinger2006} & mixed & \gls{mos},\gls{idw} & Germany & 24-120h & solar power\\
\tcite{Haben2018} & station-based & \gls{kde},\gls{sslr},\gls{arwd},\newline\gls{arwdy},\gls{hwt} & United Kingdom & up to 4 days & low voltage electricity load\\
\tcite{Alessandrini2015} & station-based & \gls{anen} & Sicily & 0-132h & wind power\\\Xhline{2\arrayrulewidth}
\tcite{Sperati2016} & grid-based & \gls{pdf},\gls{nn},\gls{vd},\gls{emos},\gls{pe} & Italy & 0-72h & solar power\\
\tcite{DeFelice2015} & grid-based & \gls{lr},\gls{svm} & Italy & 1-2 months & electricity demand\\
\tcite{Davo2016} & grid-based & \gls{pca},\gls{anen},\gls{nn} & Sicily & 0-72h & wind power,solar radiation\\\Xhline{2\arrayrulewidth}
\textbf{This thesis} & grid-based & \gls{lr},\gls{arma},\gls{armax},\gls{pca} & Germany & 1-24h & electricity load\\
%\tcite{Aertsen2012} & mixed & \gls{ok},\gls{ck},\gls{rk} & long term & TODO\\ %  maybe not relevant du to only considering station-based measures and thus using interpolation methods (kriging)
%\tcite{Kaminska-Chuchmala2014} & distributed &  & & & \\
%\tcite{Fairley2017} & grid?! & methods?! & \gls{ecmwf} & none?! & TODO\\ % no forecast
%\tcite{Voivontas1998} & grid?! & & \gls{sdhws} & none?! & TODO\\
\end{tabularx}
\caption{List of related works regarding the type of the used weather data, used methods, place of origin of the data, forecast horizon and forecast time series.}
\label{tab:relwork}
\end{sidewaystable}

