\chapter{Related Work}
\label{ch:RW}

In this chapter, the subject of this thesis will be compared to similar works, a few points will be outlined and considered to be either valuable in terms of relevance for this thesis. There will also be a few points about the process of research.\\

\section{Research}
\label{sec:res}

Gathering information is a key element in research. Therefor, \tcite{arXiv},\tcite{scholGoogle} and \tcite{base} have been used in order to find suitable reading.\\
As the title of a work often gives a good overview, a major criteria at searching for similar papers was if the title implies working with geographic or grid-based data and/or has an application in the field of energy networks and aims at forecasting correlating values.\\
Another criteria is the abstract and/or introduction, where a priority was to check wether the used data is grid based and if not directly mentioned in the title, if or how forecasting is done.\\
%This chapter is supposed to summarise previous work of other researchers related to your topic.
%The aim is to give an overview of existing literature while highlighting differences and similarities to this thesis.
%Please choose a coherent citation style throughout the thesis. For example
%\begin{itemize}
%	\item Direct citation of results, an approach or similar
%	%\item[] \Textcite{Fan.2015} find that their method improves the benchmark.
%	\item Indirect citation
%	\item[] Recent research highlights the importance of this method %\Parencite{Fan.2015}.
%	\item Direct citation
%	\item[] \textquote{\emph{Energy optimisation in buildings is important}} %\Parencite{Fan.2015}.
%\end{itemize}

\section{Related Work}
\label{sec:rw}

When it comes to weather based prediction of power, a lot of papers have been puplished. Some of them also used data from \gls{ecmwf}, but in the process of research, there didn't come up any that had a focus on the aspect of grid-based data.\\

As, according to \tcite{Li2009}, especially temperature and perceived temperature have a great impact on energy demand, a lot of the papers that combine forecasting energy demand or generation with using weather data are focusing either on forecasting \gls{pv} electricity generation as in \tcite{Bofinger2006} and \tcite{Sperati2016} or on electricity generation from wind as in \tcite{Davo2016} and \tcite{Alessandrini2015}.\\

In the following, a few papers will be outlined and explained regarding their type of data, used forecasting methods, the forecast horizon and their forcasted timeseries.\\

For time series forecasting, often used methods are e.g.\ \gls{arma} models as mentioned in \tcite{Hyndman2018}, but  also \gls{nn}, where it is common to reduce the number input variables in order to speed up computation, which is desirable for the huge amount of grid-based data that grows quadratically with size. Last there are also some papers that use regression models other than \gls{arma} such as simple \gls{lr},\gls{mlr} or \gls{svm}.\\\
E.g.\ \tcite{Aguiar2016} uses \gls{ann} to do intra-day forecasting of solar radiation within 1-6h on Gran Canaria and, as in this thesis, data from \gls{ecmwf} is used.\\
\tcite{Alessandrini2015} proposes a novelty by applying an \gls{anen} method to retain a probabilistic wind power forecast. Here the data from \gls{ecmwf} is indirectly used by feeding it into the \gls{rams} to get forecast data for the prediction. The temporal forecast horizon in this case is also rather short with 0-132h of probabilistic forecasting.\\
\tcite{Bofinger2006} also used data from \gls{ecmwf}, but also data from local weather stations to forecast solar power output within a 24-120h range. The data is then refined using \gls{mos} and \gls{idw}, spatially interpolated and then simulated for germany.\\
Another paper that used data from \gls{ecmwf} is \tcite{Davo2016}. Here also data from \gls{noaaesrl} was used which was provided for an online competition hosted by \tcite{kaggle}. Reference power data was obtained from \tcite{terna}. This is the only paper so far using \gls{pca} to reduce dimensionality, but as in \tcite{Alessandrini2015}, \gls{anen} is used, whereas here, \gls{nn} are used before. The target values here are both solar radiation and wind power produced over Sicily within a 0-72h range.\\
Similar to this thesis, \tcite{DeFelice2015} aims to forecast electricity demand using data from \gls{ecmwf}, though  for italy and with a medium-term temporal range of around 1-2 months in contrast to the short-term range targeted in this thesis. Therefor \gls{lr} and \gls{svm} are used. As power prediction is a rather complex problem, it is not very surprising that the non-linear \gls{svm} performs better than \gls{lr}.\\
In terms of this thesis, a very interesting paper is \tcite{Diagne2013} where different forecasting methods are reviewed, even though for solar radiation forecasting. Also different data sources are compared, specifically \gls{ecmwf},\gls{mm5} and \gls{wrf}. The paper focuses on \gls{ar} methods including \gls{arma},\gls{arima} and \gls{cards} and \gls{nn} with \gls{ann} and \gls{wnn} considering short time ranges from 5 min up to 6h.\\
In contrast to that, \tcite{Ludwig2015} does not consider \gls{nn}, but therefor compares \gls{lasso} and \gls{rf} next to \gls{arma} and \gls{armax} models. The target value here is the german electricity price for the next day, thus having a 24h forecast horizon, and the used data is obtained by distributed measures from \gls{dwd} for weather data and from \gls{epex} for the price history. A desirable side effect from \gls{rf} is the output of the variable importance which is useful in order to filter variables by order of their importance.\\
An interesting work about low voltage load forecasting is \tcite{Haben2018} where \gls{kde},\gls{sslr},\gls{arwd},\gls{arwdy} and \gls{hwt} are comepared for a forecast horizon of up to 4 days. The weather data used here to refine the forecasting results are station-based.\\
Further methods are presented in \tcite{Salcedo-Sanz2018} with combinations of \gls{cro},\gls{elm},\gls{gga},\gls{mars},\gls{svr} for short-term solar radiation forecast of 24h in Australia. The used data comes mostly from \gls{ecmwf}, but also from \gls{silo} and thus uses gridded as well as non-gridded data.\\
Another application of \gls{ecmwf} data is proposed in \tcite{Sperati2016} for short-term solar power forecasting within a 0-72h time range using a \gls{pdf} combined with \gls{nn},\gls{vd},\gls{emos} and \gls{pe}.\\

%Other works regarding \gls{arima} models such as \tcite{Kaminska-Chuchmala2014} are also considered valuable information sources, though using these methods e.g.\ in this case to forecast internet traffic load which is quite similar to power demand, as the internet as well as the german power network both are supposed to work bidirectional with future regard.\\
Of course there where other papers considered such as \tcite{Kaminska-Chuchmala2014} in particular due to its subject of forecasting internet traffic load and the high correlation between internet traffic load and electricity load \Parencite{Morley2018}. The use of \gls{ok} is also interesting, but not used in this thesis, which is why this subject has not been considered for further research. Another great example would be \tcite{Fairley2017} where spatio-temporal variation in wave power and its implications for electricity supply are being discussed which combines localization issues and the electricity network, but unfortunately in this case the aim is not to forecast but only to examine the problem.\\

\Cref{tab:relwork} provides an overview about some of the mentioned related works including further information in terms of spatial distribution of the used data that does not correspond to the target value, used methods, origin of the data, temporal scope and the target value.\\
%It is to mention that regarding the temporal scope, short term means up to a few days, middle term refers to up to a few months and long term is about seasonal forecasting which possibly includes multiple years.\\

% TODO add location column?!
\begin{sidewaystable}[!ht]%
\rowcolors{2}{white}{gray!25}
\centering
\footnotesize
\begin{tabularx}{\linewidth}{llLlL}
\tablehead paper & \tablehead type of data & \tablehead methods & \tablehead forecast horizon & \tablehead forecasted timeseries \\\hline
\tcite{Aguiar2016} & grid-based & \gls{nn} & 1-6h & solar radiation\\
\tcite{Alessandrini2015} & station-based & \gls{anen} & 0-132h & wind power\\
\tcite{Bofinger2006} & mixed & \gls{mos},\gls{idw} & 24-120h & solar power\\
\tcite{Davo2016} & grid-based & \gls{pca},\gls{anen},\gls{rams} & 0-72h & wind power,solar radiation\\
\tcite{DeFelice2015} & grid-based & \gls{lr},\gls{svm} & 1-2 months & electricity demand\\
\tcite{Diagne2013} & grid-based & \gls{arma},\gls{arima},\gls{cards},\gls{ann},\gls{wnn} & 5 min-6h & solar radiation\\
\tcite{Haben2018} & station-based & \gls{kde},\gls{sslr},\gls{arwd},\gls{arwdy},\gls{hwt} & up to 4 days & low voltage electricity load\\
\tcite{Ludwig2015} & station-based & \gls{arma},\gls{armax},\gls{lasso},\gls{rf} & 24h & energy prices\\
\tcite{Salcedo-Sanz2018} & mixed & \gls{elm},\gls{cro},\gls{mars},\gls{mlr},\gls{svr},\gls{gga} & 24h & solar radiation\\
\tcite{Sperati2016} & grid-based & \gls{pdf},\gls{nn},\gls{vd},\gls{emos},\gls{pe} & 0-72h & solar power\\
%\tcite{Aertsen2012} & mixed & \gls{ok},\gls{ck},\gls{rk} & long term & TODO\\ %  maybe not relevant du to only considerinng station-based measures and thus using interpolation methods (kriging)
%\tcite{Kaminska-Chuchmala2014} & distributed &  & & & \\
%\tcite{Fairley2017} & grid?! & methods?! & \gls{ecmwf} & none?! & TODO\\ % no forecast
%\tcite{Voivontas1998} & grid?! & & \gls{sdhws} & none?! & TODO\\
\end{tabularx}
\caption{List of related works and used methods respectively as well as some further details.}
\label{tab:relwork}
\end{sidewaystable}

One key difference of the presented works to this thesis is that here, reanalyzed data from \gls{ecmwf} is used for prediction which means, that the forecasts might behave differently from forecasts in other works. This also means that results from this thesis possibly won't exactly match results using the same procedure with real-time data.\\

