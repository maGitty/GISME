\chapter{Related Work}
\label{ch:RW}

This chapter gives an overview of related work in the field of energy forecasting considering grid-based data. After describing the general approach of search, there will be some sources ordered by degree of relation to this thesis.\\

In order to find relevant literature, \tcite{arXiv},\tcite{scholGoogle} and \tcite{base} have been used.\\
In the process of search, the following criteria have been used to identify relevant literature:
\begin{itemize}
  \item The title suggests that the authors work with geographic or grid-based data
  \item The title implies that the subject is being situated in the field of energy networks
  \item The title suggests that the authors aim at forecasting
  \item The abstract or introduction suggests that the authors work with geographic or grid-based data
  \item The abstract or introduction suggests that the authors aim at forecasting or rather how forecasting is done
\end{itemize}

When it comes to weather based prediction of electricity generation, a lot of papers have been published. Some of them use grid-based data, but in the process of search, there did not come up any that had a focus on the aspect of how to most suitably include grid-based data to forecast energy time-series. However, there has been related work, even though not having the exact same subject, which is evaluating how to properly include grid-based data for energy time series forecasting. As a result, these papers are employed as reference knowledge. Subsequently, they will be outlined and explained regarding their type of data, the forecast time series, applied forecasting methods and the forecast horizon.\\

%1 Davò, grid-based, output: wind power, solar irradiance, methods: pca,nn,anen, horizon: 0-72h, Sicily(Italy)
As the related works are presented ordered by relevance, the first papers are quite close to this thesis, forecasting electricity generation with grid-based data.\\
The first paper presented is \tcite{Davo2016} who utilize grid-based wind speed data generated by applying the \gls{rams} with boundary conditions from \gls{ecmwf}. Furthermore, they acquire grid-based data of solar radiation energy per square meter as one of the two forecast time series is the solar irradiance. The data is coming from \gls{noaaesrl} and was provided for an online competition hosted by \tcite{kaggle}. Reference power data is obtained from \tcite{terna} as the other predicted time series is the wind power produced over Sicily. A \gls{pca} is employed which is an important point as grid-based data is even more prone to the curse of dimensionality because of the two additional dimensions. In terms of forecasting they apply \gls{nn} and an \gls{anen}. The forecast horizon has a range of 0 to 72 hours and the output is a prediction of both, wind power and solar radiation.\\
%2 De Felice, grid-based, output: electricity demand in Italy, methods: lr,svm, horizon: 1-2 months, Italy
Similar to this thesis, \tcite{DeFelice2015} use grid-based data from \gls{ecmwf} to forecast the electricity demand, though for Italy. \gls{lr} and a \gls{svm} are applied. Given that power prediction is a rather complex problem, the non-linear \gls{svm} performs better than a simple \gls{lr}.\\
%3 Sperati, grid-based, output: solar power in Italy, methods: pdf,nn,vd,emos,pe, horizon: 0-72h, Italy
Another application of grid-based data from \gls{ecmwf} is proposed in \tcite{Sperati2016} for solar power prediction in Italy. They implement a \gls{pdf} combined with \gls{nn},~\gls{vd},~\gls{emos} and \gls{pe}. The time series forecast includes a range of 0-72 hours.\\
%4 Alessandrini, station-based, output: wind power, methods: anen, horizon: 0-132h, Sicily(Italy)
The second group of related works covers those that also forecast electricity generation, but in contrast to the first group, make use of station-based data.\\
\tcite{Alessandrini2015} utilize non-gridded wind and power data from a wind farm in northern Sicily in Italy with which they forecast generated wind power. Here, a novel approach, an \gls{anen}, which originally is used for meteorological ensemble forecasts, is applied to the data to retain a probabilistic prediction for the next 0-132 hours.\\
%5 Haben station-based, output: low voltage load, methods: kde,sslr,arwd,arwdy,hwt, horizon: up to 4 days, UK
A work where station-based weather data is applied to forecast low voltage load in the United Kingdom, has been published by \tcite{Haben2018}. They implement \gls{kde}, \gls{sslr}, \gls{arwd}, \gls{arwdy} and \gls{hwt} and compare them for forecast horizons of up to 4 days.\\
%6 Bofinger station-based, output: solar power generation, methods: mos,idw, horizon: 24-120h, Germany
\tcite{Bofinger2006} also acquire data from local weather stations to forecast solar power generation. The data is then refined with grid-based data from \gls{ecmwf} by applying \gls{mos} and \gls{idw}, spatially interpolated and then simulated for Germany in order to predict a temporal range of 24-120 hours.\\
%7 Aguiar mixed, output: ghi, methods: nn, horizon: 1-6h, Gran Canaria Island
In addition, there is \eg \tcite{Aguiar2016}, using both, grid-based and station-based weather data to improve \gls{ghi} forecasts on Gran Canaria Island. In order to achieve this, \gls{nn} are applied. As the authors consider intra-day forecasting, the forecast horizon here is limited to a range of 1 to 6 hours.\\
%8 Diagne grid-based, output: solar radiation, methods: arma,arima,cards,nn,wnn, horizon: 5min-6h, no location?!
The last group of related works deals with those that either utilize grid-based, station-based or both types of data to forecast various time series.\\
In terms of this thesis, a very interesting paper is \tcite{Diagne2013}, where grid-based weather data is used for solar radiation forecasting, which is similar to \gls{ghi} which is forecast in the previous paper. Also different data sources are compared, specifically \gls{ecmwf}, \gls{mm5} and \gls{wrf}. The paper focuses on \gls{ar} methods including \gls{arma}, \gls{arima} and \gls{cards}, \gls{nn} and \gls{wnn} considering short time ranges from 5 min up to 6h.\\
%9 Salcedo grid-based, output: solar radiation, methods: cro,elm,gga,mars,svr, horizon: 24h, Australia
Similarly, \tcite{Salcedo-Sanz2018} also utilize grid-based weather data to forecast solar radiation in Australia. The evaluated methods are combinations of \gls{cro}, \gls{elm}, \gls{gga}, \gls{mars}, \gls{svr} for a forecast horizon of 24 hours.\\
%10 Ludwig station-based, output: electricity price, methods: arma,armax,rf,lasso, horizon 24h, Germany
A different paper from \tcite{Ludwig2015}, investigates the usage of station-based weather data from \gls{dwd} for electricity price forecasting in Germany. The price history is obtained from \gls{epex}. The work does not consider \gls{nn}, but rather compares \gls{lasso} and \gls{rf} in addition to \gls{arma} and \gls{armax} models. A desirable side effect from \gls{rf} is the output of the variable importance which is useful in order to filter variables by order of their relevance. As a this work has a focus on short-term forecasts, the forecast time series here is the electricity price for the next day, thus having a forecast horizon of 24 hours.\\

There where other papers considered such as \tcite{Kaminska-Chuchmala2014}, in particular due to its subject of forecasting internet traffic load and the high correlation between internet traffic load and electricity load, as can be read in \tcite{Morley2018}. They apply \gls{ok} to spatially interpolate station-based data. As interpolation itself is not topic of this thesis, this subject has not been considered for further research. However, due to the great similarity between these fields, it is an has great for related research and also influenced this work. Furthermore it underlines the benefit of grid-based data by illustrating the effort processing station-based data which is not neccessary for grid-based data. Another great example utilizing station-based data is \tcite{Fairley2017} investigating marine energy generation evaluating implications for electricity supply are being discussed. This combines localization issues and the electricity network. Unfortunately, the aim here is not to forecast but only to examine the problem.\\

\Cref{tab:relwork} provides an overview about the mentioned related works regarding the type of the used weather data, used methods, the place of origin of the data, the forecast horizon and the forecast time series.\\

% TODO add location column?!
\begin{sidewaystable}[!ht]%
\rowcolors{2}{white}{gray!25}
\centering
\footnotesize
\begin{tabularx}{\linewidth}{llXlll}
\tablehead paper & \tablehead type of weather data & \tablehead methods & \tablehead location & \tablehead forecast horizon & \tablehead forecast time series \\\Xhline{2\arrayrulewidth}
\tcite{Davo2016} & grid-based & \gls{pca},\gls{anen},\gls{nn} & Sicily & 0-72h & wind power,solar radiation\\
\tcite{DeFelice2015} & grid-based & \gls{lr},\gls{svm} & Italy & 1-2 months & electricity demand\\
\tcite{Sperati2016} & grid-based & \gls{pdf},\gls{nn},\gls{vd},\gls{emos},\gls{pe} & Italy & 0-72h & solar power\\\Xhline{2\arrayrulewidth}
\tcite{Alessandrini2015} & station-based & \gls{anen} & Sicily & 0-132h & wind power\\
\tcite{Haben2018} & station-based & \gls{kde},\gls{sslr},\gls{arwd},\newline\gls{arwdy},\gls{hwt} & United Kingdom & up to 4 days & low voltage electricity load\\
\tcite{Bofinger2006} & mixed & \gls{mos},\gls{idw} & Germany & 24-120h & solar power\\
\tcite{Aguiar2016} & mixed & \gls{nn} & Gran Canaria Island & 1-6h & solar radiation\\\Xhline{2\arrayrulewidth}
\tcite{Diagne2013} & grid-based & \gls{arma},\gls{arima},\gls{cards},\newline\gls{nn},\gls{wnn} & - & 5 min-6h & solar radiation\\
\tcite{Salcedo-Sanz2018} & grid-based & \gls{elm},\gls{cro},\gls{mars},\newline\gls{mlr},\gls{svr},\gls{gga} & Australia & 24h & solar radiation\\
\tcite{Ludwig2015} & station-based & \gls{arma},\gls{armax},\gls{lasso},\gls{rf} & Germany & 24h & energy prices\\\Xhline{2\arrayrulewidth}
\textbf{This thesis} & grid-based & \gls{lr},\gls{arma},\gls{armax},\gls{pca} & Germany & 1-24h & electricity load\\
%\tcite{Aertsen2012} & mixed & \gls{ok},\gls{ck},\gls{rk} & long term & TODO\\ %  maybe not relevant du to only considering station-based measures and thus using interpolation methods (kriging)
%\tcite{Kaminska-Chuchmala2014} & distributed &  & & & \\
%\tcite{Fairley2017} & grid?! & methods?! & \gls{ecmwf} & none?! & TODO\\ % no forecast
%\tcite{Voivontas1998} & grid?! & & \gls{sdhws} & none?! & TODO\\
\end{tabularx}
\caption{List of related works regarding the type of the used weather data, used methods, place of origin of the data, forecast horizon and forecast time series.}
\label{tab:relwork}
\end{sidewaystable}

