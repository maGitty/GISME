\pdfbookmark[1]{Abstract}{Abstract}

\chapter*{Abstract}

\begin{center}
  \begin{minipage}{12cm}
    \begin{sloppypar}
    	There is no doubt, that electricity demand depends more and more on local weather as the share of renewable resources in electricity generation grows. This is, why it is important to find new and more reliable ways to react on the fluctuations that become increasingly volatile. One possible way is grid-based data, such as that from the \gls{ecmwf}, which is rectified based on past measurement errors and therefore may be even more correct than actual measures from local weather stations. However, a result of this work is that including single grid points may not lead to an improvement. An aggregation of points over an area, however, may do so. This gives new insight about how to best use grid-based data in order to use weather data to improve forecasts in the energy sector.
%    	Whilst the energy transition in Germany proceeds, it gets more and more important to implement techniques, allowing to react rapidly on fluctuations in energy demand. This is important, because those fluctuations become more volatile as the dependence on weather grows. Therefore, 
%    	This is an abstract. It is fine because it is small and nice.
%		As the share of electricity from regenerative sources is growing constantly, the weather becomes an increasingly important factor in the analysis of electricity markets. Hence, this thesis uses local weather data to predict electricity spot prices. More precisely, we include wind speed and temperature from individual German weather stations into time series and statistical learning models. However, as the available weather information is vast and renewable power is not generated everywhere, we use random forests and Bayesian structural time series to perform a feature selection. Overall, we manage to improve our forecasting accuracy of the EPEX electricity prices by up to \SI{7.69}{\percent} in terms of root mean squared error and up to \SI{8.19}{\percent} in terms of mean absolute error.
    \end{sloppypar}
  \end{minipage}
\end{center}