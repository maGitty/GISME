\chapter{Evaluation}
\label{ch:Evaluation}

\section{Data}

Describe the data set you are using. Use appropriate visualization (\eg graphs, statistical summaries \etc) to help the reader get to know your data set.

\Cref{tab:example} is an example table. Remember to use full sentences in your caption and explain everything one can see in the table there as well. You can of course also use a simpler format for your table.

\begin{table}[h!]%
\caption{Example table with rotated table heads to save space and two different row colours to ease the readability.}
\rowcolors{2}{gray!25}{white}
\centering
\footnotesize
\begin{tabular}{lll}
\toprule \noalign{\smallskip}
\rottblhead{\tablehead Header 1} & \rottblhead{\tablehead Header 2} & \rottblhead{\tablehead Header 3} \\ \midrule
entry 1 & entry 2 & entry 3 \\ 
entry 1 & entry 2 & entry 3 \\ 
entry 1 & entry 2 & entry 3 \\ 	\bottomrule
\end{tabular}
\label{tab:example}
\end{table}

\section{Results}

Describe the results you have obtained using your methods described above. Again use proper visualization methods.

\subsection{Experiment 1}

\dots

\subsection{Experiment 2}

\dots