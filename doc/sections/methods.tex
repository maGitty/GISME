\chapter{Methodology}
\label{ch:methods}

%This chapter should introduce to the theoretical background of your thesis. Any method you use to obtain the results later should be introduced and explained. 
In this chapter, the used forecasting methods are explained further and why they were used, but also other methodological aspects of this thesis will be outlined.\\

%Using weather data from ECMWF Copernicus Climate Change Service (C3S).\\
%Using load data from \url{https://data.open-power-system-data.org/}.\\
%First downloaded whole Datasets from 2006-2019, but as the load for germany is properly available since 2015, now reduced dataset to 2015-2019.\\
%Also checked for non-existing values, only 2 last timestamps values for the load are missing.\\

\section{Data acquisition}
\label{sec:dataq}

In order to acquire the needed weather data, \gls{ecmwf}'s Python-API is used for automated data acquisition. The API has been extended by some functionality recently and allows to download the data with different extensions. The chosen extension is .nc, because Python's xarray library allows performant access to these files.\\

\section{Forecasting methods}
\label{sec:forecastmet}

For time series forecasting, often used methods are \eg \gls{arma} models as mentioned in \tcite{Hyndman2018}, but  also \gls{nn}, where it is common to reduce the number of input variables in order to speed up computation, which is desirable for the huge amount of grid-based data that grows quadratically with size. There are also some papers that use regression models other than \gls{arma} such as \gls{lr}, \gls{mlr} or \gls{svm}.\\
\tcite{Aguiar2016} applies \gls{nn} to do intra-day solar radiation forecasting with a forecasting horizon of 1-6 hours on Gran Canaria and, as in this thesis, grid-based data from \gls{ecmwf} is used.\\

%\subsection{Linear Regression}
\subsection{ARMA}

One of the most frequently used methods for forecasting in time series forecasting is \gls{arma}, which is a combination of \gls{ar} and \gls{ma} terms. \gls{arima} is quite similar to this, but involves an additional differentiation term to consider the trend of past data. The formal description of an \gls{arma} is given as follows:\\

\begin{equation}
y_t = c+\epsilon_t+\sum_{i=1}^{p}a_iy_{t-i}+\sum_{j=1}^{q}b_j\epsilon_{t-j}
\label{eq:arma}
\end{equation}

With $c$ as a constant, $\epsilon_t$ as noise terms with respect to time $t$, $p$ as size of the \gls{ar} part, $q$ as size of the \gls{ma} part, $a$ and $b$ for the \gls{ar} and \gls{ma} coefficients respectively and $y_t$ as the predicted value.\\

\subsection{ARMAX}

Another similar method is \gls{armax}, which includes an additional term for exogenous variables. This term can be used to include relations to external factors that do not depend on the endogenous data. The formal description is given by:\\

\begin{equation}
y_t = c+\epsilon_t+\sum_{i=1}^{p}a_iy_{t-i}+\sum_{j=1}^{q}b_j\epsilon_{t-j}+\sum_{k=1}^{n}c_kx_k
\label{eq:armax}
\end{equation}

Which almost equals to \Cref{eq:arma} for the \gls{arma}. There is an additional term for $n$ included exogenous variables $x$ with $c$ as the respective coefficients.

%\subsubsection{only calendar variables as exogenous inputs}
%
%\subsubsection{additional weather variables as exogenous inputs}

\section{Feature Selection}
\label{sec:featsel}

\subsection{Principal Component Analysis}

\section{Forecast Evaluation}
\label{sec:fceval}

\subsection{Root Mean Squared Error}

\begin{equation}
RMSE = \sqrt{\frac{1}{n} \sum_{i=1}^{n} (a_i-p_i)^2}
\label{eq:rmse}
\end{equation}

With $n$ for the size, a for the actual values and p for predicted values.\\

\subsection{Mean Absolute Percentage Error}

\begin{equation}
MAPE = \frac{1}{n}\times 100 \sum_{i=1}^{n} \left|\frac{a_i-p_i}{a_i}\right|
\label{eq:mape}
\end{equation}

With $n$ for the size, a for the actual values and p for predicted values.\\

%Maybe use Random Forests for variable selection as in Nicoles paper? \Parencite{Ludwig2015}\\

%This is an example for a simple equation without equation numbering.
%$$
%\sum\limits_{i=1}^{n}{x_i}
%$$

%You can also use equation numbering if you need to refer to an equation later \eg \Cref{eq:ex1}.


%\begin{equation}
%a^2 + b^2 = c^2
%\label{eq:ex1}
%\end{equation}
%
%Additionally, simple equations can be put inline with the text, for example, $x \in X$. Remember to set all variables in math font \ie all $x$, $i$ and so on.
%
%\section{Method 2}
%
%\dots

