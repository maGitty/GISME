\chapter{Methodology}
\label{ch:methods}

%This chapter should introduce to the theoretical background of your thesis. Any method you use to obtain the results later should be introduced and explained. 
This chapter introduces the methods that are applied in this thesis. The first group of methods comprises the used forecasting methods, the second group involves feature selection techniques and the last group covers methods that are used to evaluate forecasts.\\
%This chapter introduces the methods that are applied in this thesis. First, the used forecasting methods are introduced. After that, the used feature selection techniques are explained and also the methods that were applied for error measuring.\\
%In this chapter, the used forecasting methods are explained and why they were used. After that, other methodological aspects of this thesis will be outlined.\\

%Using weather data from ECMWF Copernicus Climate Change Service (C3S).\\
%Using load data from \url{https://data.open-power-system-data.org/}.\\
%First downloaded whole Datasets from 2006-2019, but as the load for germany is properly available since 2015, now reduced dataset to 2015-2019.\\
%Also checked for non-existing values, only 2 last timestamps values for the load are missing.\\

%\section{Data acquisition}
%\label{sec:dataq}
%
%In order to acquire the needed weather data, \gls{ecmwf}'s Python-API is used for automated data acquisition. The API has been extended by some functionality recently and allows to download the data with different extensions. The chosen extension is .nc, because Python's xarray library allows performant access to these files.\\
%
%The used load data has been downloaded from Open Power System Data\footnote{\url{https://data.open-power-system-data.org/time_series/}}. As the format of the downloaded data is .csv, it has also been converted to .nc to obtain performant and uniform access to all data.\\

\section{Forecasting methods}
\label{sec:forecastmet}

For time series forecasting, often used methods are \eg \gls{arma} models as mentioned in \tcite{Hyndman2018}. This sections will introduce the methods that are applied in this thesis to forecast the electricity load.\\% Forecasting: Principles and Practice\footnote{\url{https://otexts.com/fpp2}}.%, but  also \gls{nn}, where it is common to reduce the number of input variables in order to speed up computation, which is desirable for the huge amount of grid-based data that grows quadratically with size. There are also some papers that use regression models other than \gls{arma} such as \gls{lr}, \gls{mlr} or \gls{svm}. \tcite{Aguiar2016} applies \gls{nn} to do intra-day solar radiation forecasting with a forecasting horizon of 1-6 hours on Gran Canaria and, as in this thesis, grid-based data from \gls{ecmwf} is used.\\

%\subsection{Linear Regression}
\subsection*{ARMA}

The \gls{arma} model is a combination of \acrfull{ar} and \gls{ma} terms. The formal description is given by\\

\begin{equation}
y_t = c+\sum_{i=1}^{p}\phi_iy_{t-i}+\sum_{j=1}^{q}\rho_j\epsilon_{t-j}+\epsilon_t~,
\label{eq:arma}
\end{equation}

with $c$ as a constant, $\epsilon_t$ as noise term with respect to time $t$, $p$ as size of the \gls{ar} part, $q$ as size of the \gls{ma} part, $\phi$ and $\rho$ for the \gls{ar} and \gls{ma} coefficients respectively and $y_t$ as the response variable.\\

\subsection*{ARMAX}

An extension of \gls{arma} is \gls{armax}, which includes an additional term for exogenous variables. This term can be used to include relations to external factors that do not depend on the endogenous data. It is formally described as\\

\begin{equation}
y_t = c+\sum_{i=1}^{p}\phi_iy_{t-i}+\sum_{j=1}^{q}\rho_j\epsilon_{t-j}+\sum_{k=1}^{n}\eta_kx_k+\epsilon_t~.
\label{eq:armax}
\end{equation}

The only difference between \Cref{eq:armax} and \Cref{eq:arma} is the additional term $\sum_{k=1}^{n}\eta_kx_k$ for the \gls{armax} for $n$ included exogenous variables $x$ with $\eta$ as the respective coefficients.\\
%\Cref{eq:armax} almost equals \Cref{eq:arma} for the \gls{arma} model, but here, there is an additional term $\sum_{k=1}^{n}\eta_kx_k$ for $n$ included exogenous variables $x$ with $\eta$ as the respective coefficients.

%\subsubsection{only calendar variables as exogenous inputs}
%
%\subsubsection{additional weather variables as exogenous inputs}

\section{Feature selection techniques}
\label{sec:featsel}

Because the used weather data is grid-based, there are two more dimensions than usual, where only one value per time step exists for a variable. This is why feature selection here is more important in order to obtain a reasonable computation time. In the following, the used methods for feature selection are presented.\\

\subsection*{Naive approach}

First, naive techniques are presented, that are used to reduce the huge amount of grid-based weather data. They are reduced along the two spatial dimensions, longitude and latitude, for each step in time, respectively. These are simple functions such as the maximum or the mean. An exemplary formula for reducing the data along longitude and latitude using the mean is given as\\

\begin{equation}
x_t = \frac{1}{l \times m} \sum_{i=1}^{l}\sum_{j=1}^{m}x_{ij}~,
\end{equation}

where $x_t$ is the calculated mean for time $t$, $l$ and $m$ are the size of the data along the axis of the longitude and latitude and $x_{ij}$ is the value of a weather variable at the grid point with longitude $i$ and latitude $j$.

\subsection*{Using population data}

Another method involves population data from Eurostat\footnote{\url{https://ec.europa.eu/eurostat/data/database}}. It contains the population of NUTS 3 level regions. The regions are sorted by population and those with the highest population are used to filter the respective grid points that are then used as exogenous variables.\\

%\subsection{Principal Component Analysis}

\section{Forecast Evaluation}
\label{sec:fceval}

In order to estimate whether the used model performs well, it is important to apply suitable metrics to evaluate the results. In the following, the four used metrics are introduced, where for each metric, $k$ is the number of forecast values, $y$ the actual values and $\hat{y}$ the predicted values.\\

\subsection*{Root Mean Squared Error}

The first metric is the \gls{rmse}, which is an often used, scale-dependent accuracy measure that calculates the root of the squared mean of the differences between the forecast and the actual values. It is described by\\

\begin{equation}
RMSE = \sqrt{\frac{1}{k} \sum_{i=1}^{k} (y_i-\hat{y}_i)^2}~.
\label{eq:rmse}
\end{equation}

\subsection*{Mean Absolute Error}

The second metric is the \gls{mae}, which is another scale-dependent accuracy measure that averages absolute errors. The equation for the \gls{mae} is described by\\

\begin{equation}
MAE = \frac{1}{k} \sum_{i=1}^{k} \left|y_i-\hat{y}_i\right|~.
\end{equation}

\subsection*{Mean Percentage Error}

The third metric is the \gls{mpe}, which is a relative measure of the prediction accuracy. Since it is multiplied by 100 after dividing it by the size of the predictions, it is called a percentage error. The equation is\\
%The \gls{mpe} is the computed average of percentage errors by which forecasts of a model differ from actual values of the quantity being forecast

\begin{equation}
MPE = \frac{100}{k} \sum_{i=1}^{k} \frac{y_i-\hat{y}_i}{y_i}~.
\end{equation}

\subsection*{Mean Absolute Percentage Error}

The fourth metric is the \gls{mape}, which is similar to the \gls{mpe}, but takes the absolute value of each single error instead. The equation for the \gls{mape} is given by\\

\begin{equation}
MAPE = \frac{1}{k}\times 100 \sum_{i=1}^{k} \left|\frac{y_i-\hat{y}_i}{y_i}\right|~.
\label{eq:mape}
\end{equation}


%Maybe use Random Forests for variable selection as in Nicoles paper? \Parencite{Ludwig2015}\\

%This is an example for a simple equation without equation numbering.
%$$
%\sum\limits_{i=1}^{n}{x_i}
%$$

%You can also use equation numbering if you need to refer to an equation later \eg \Cref{eq:ex1}.


%\begin{equation}
%a^2 + b^2 = c^2
%\label{eq:ex1}
%\end{equation}
%
%Additionally, simple equations can be put inline with the text, for example, $x \in X$. Remember to set all variables in math font \ie all $x$, $i$ and so on.
%
%\section{Method 2}
%
%\dots

