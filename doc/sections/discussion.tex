\chapter{Discussion}
\label{ch:discussion}

Considering the main research question of this thesis, whether grid-based weather information does have a benefit on energy forecasting, it is now possible to approach an answer. As already pointed out in \Cref{sec:results}, using the actual grid points themselves does not result in any improvement, but rather causes the results to deteriorate. This seems to suggest that there is no benefit at all in using grid-based data for energy forecasting. However, grid-based data can be very useful when it is being conglomerated or compressed in a representative form, such as the mean over the longitude and latitude. This can also be observed in \Cref{sec:results}, where the averaged 2 metre temperature actually improves the forecast accuracy. The special upside in using grid-based data, such as the data from \gls{ecmwf}, is, that it can be conglomerated over any desired locality as it is available for most locations. This allows it to be used for forecasts at arbitrary locations. Even though, from the missing results of most of the given weather variables, it can not be said which of them are most suitable to be used as inputs for energy forecasting. Also, there was too much effort invested in analysing the used data which was not necessary. The spent time would have better been used to examine further model sizes or checking which weather variables improve the forecast accuracy. Another critical point is the missing of alternative methods. In this thesis, only \gls{arma} and \gls{armax} models have been used, but it is still unclear if there are better methods for energy related time series forecasting.\\
%outlook: try other methods to compress the data as representative as possible, mean over highest populated regions, or regions filtered by economic activity also compressed or so
%Looking at the results from \Cref{sec:results}, 

%This chapter is supposed to discuss your results. Point out what your results mean.
%What are the limitations of your approach, managerial implications or future impact?
%
%Explain the broader picture but be critical with your methods.